\documentclass[12pt,a4paper]{article}
\usepackage[utf8]{inputenc}
\usepackage[T1]{fontenc}
\usepackage{amsmath, amssymb, amsthm}
\usepackage{geometry}
\usepackage[framemethod=tikz]{mdframed}
\geometry{margin=1in}

% Eliminar sangría de párrafos
\setlength{\parindent}{0pt}

\title{Procesos Estocásticos: Seccion 1}
\author{Alejandro Daniel José Gómez Flórez}
\date{}

% Macro para teoremas
\newmdenv[
    backgroundcolor=green!5,
    linecolor=green!50,
    linewidth=1.5pt,
    roundcorner=5pt,
    innertopmargin=8pt,
    innerbottommargin=8pt,
    innerleftmargin=10pt,
    innerrightmargin=10pt,
    leftmargin=0pt,
    rightmargin=0pt
]{teoremabox}

\newcommand{\teorema}[1]{%
\begin{teoremabox}
\textbf{Teorema}: #1
\end{teoremabox}
}

\begin{document}

\maketitle

\section{Demostraciones}


\teorema{
\begin{equation*}
P^{(n)} = P^{(m)} \cdot P^{(n-m)}, \quad \text{en particular } P^{(n)} = P \cdot P \cdots P \ (\text{$n$ veces}) = P^n.
\end{equation*}}

\textbf{Demostración:}

Esta es la ecuación fundamental de Chapman-Kolmogorov que establece la propiedad multiplicativa de las matrices de transición. La demostraremos usando la definición probabilística y las propiedades de la esperanza condicional.

\textbf{Notación:} Sea $P_{ij}^{(n)}$ la probabilidad de transición del estado $i$ al estado $j$ en exactamente $n$ pasos, es decir:
\begin{equation*}
P_{ij}^{(n)} = \mathbb{P}(X_n = j \mid X_0 = i)
\end{equation*}

\textbf{Demostración de la ecuación de Chapman-Kolmogorov:}

Para demostrar que $P^{(n)} = P^{(m)} \cdot P^{(n-m)}$ donde $0 \leq m \leq n$, consideremos el elemento $(i,j)$ de ambos lados.

Lado izquierdo: $P_{ij}^{(n)} = \mathbb{P}(X_n = j \mid X_0 = i)$

Lado derecho: $[P^{(m)} \cdot P^{(n-m)}]_{ij} = \sum_{k \in S} P_{ik}^{(m)} P_{kj}^{(n-m)}$

Usando la ley de probabilidad total condicionada en el estado en el tiempo $m$:

\begin{align*}
P_{ij}^{(n)} &= \mathbb{P}(X_n = j \mid X_0 = i) \\
&= \sum_{k \in S} \mathbb{P}(X_n = j, X_m = k \mid X_0 = i) \\
&= \sum_{k \in S} \mathbb{P}(X_n = j \mid X_m = k, X_0 = i) \cdot \mathbb{P}(X_m = k \mid X_0 = i)
\end{align*}

Por la propiedad de Markov, $\mathbb{P}(X_n = j \mid X_m = k, X_0 = i) = \mathbb{P}(X_n = j \mid X_m = k)$, ya que el futuro solo depende del estado presente, no del pasado.

Por lo tanto:
\begin{align*}
P_{ij}^{(n)} &= \sum_{k \in S} \mathbb{P}(X_n = j \mid X_m = k) \cdot \mathbb{P}(X_m = k \mid X_0 = i) \\
&= \sum_{k \in S} P_{kj}^{(n-m)} \cdot P_{ik}^{(m)} \\
&= \sum_{k \in S} P_{ik}^{(m)} P_{kj}^{(n-m)}
\end{align*}

Esto demuestra que $P_{ij}^{(n)} = [P^{(m)} \cdot P^{(n-m)}]_{ij}$ para todo $i,j$, por lo que:
\begin{equation*}
P^{(n)} = P^{(m)} \cdot P^{(n-m)}
\end{equation*}

\textbf{Caso particular:} Para $P^{(n)} = P^n$, aplicamos inducción sobre $n$.

\textit{Caso base:} $n = 1$: $P^{(1)} = P = P^1$ (correcto por definición)

\textit{Paso inductivo:} Supongamos que $P^{(k)} = P^k$ para todo $k \leq n$. Entonces:
\begin{align*}
P^{(n+1)} &= P^{(n)} \cdot P^{(1)} \quad \text{(Chapman-Kolmogorov con } m = n\text{)} \\
&= P^n \cdot P \quad \text{(hipótesis inductiva)} \\
&= P^{n+1}
\end{align*}

Por inducción, $P^{(n)} = P^n$ para todo $n \geq 1$.

\qed

\textbf{Interpretación:} La ecuación de Chapman-Kolmogorov nos dice que para ir del estado $i$ al estado $j$ en $n$ pasos, podemos "hacer escala" en cualquier estado intermedio $k$ en el tiempo $m$, y la probabilidad total es la suma sobre todos los posibles estados intermedios del producto de las probabilidades de los dos segmentos del viaje.


\teorema{Para una cadena de Markov con matriz de transición $P = (P_{ij})$:
\begin{itemize}
    \item $\sum_{n=0}^{\infty} P_{ii}^{(n)} = \infty$ si, y solo si el estado $i$ es recurrente
    \item $\sum_{n=0}^{\infty} P_{ii}^{(n)} < \infty$ si, y solo si el estado $i$ es transitorio
\end{itemize}}

\textbf{Demostración:}

Demostraremos ambas direcciones de la equivalencia.

\textbf{Parte 1: ($\Rightarrow$)} Si el estado $i$ es recurrente, entonces $\sum_{n=0}^{\infty} P_{ii}^{(n)} = \infty$.

Sea $N_i$ el número de veces que la cadena visita el estado $i$. Por definición:
\begin{equation*}
N_i = \sum_{n=0}^{\infty} \mathbf{1}_{\{X_n = i\}}
\end{equation*}

Tomando esperanza condicionada en $X_0 = i$:
\begin{equation*}
\mathbb{E}_i[N_i] = \mathbb{E}_i\left[\sum_{n=0}^{\infty} \mathbf{1}_{\{X_n = i\}}\right] = \sum_{n=0}^{\infty} \mathbb{E}_i[\mathbf{1}_{\{X_n = i\}}] = \sum_{n=0}^{\infty} P_{ii}^{(n)}
\end{equation*}

Si el estado $i$ es recurrente, entonces la probabilidad de regresar a $i$ partiendo de $i$ es 1:
\begin{equation*}
f_{ii} = \mathbb{P}_i(\text{regresar a } i \text{ alguna vez}) = 1
\end{equation*}

Esto implica que $\mathbb{E}_i[N_i] = \infty$ (visitamos $i$ infinitas veces con probabilidad 1), por lo tanto:
\begin{equation*}
\sum_{n=0}^{\infty} P_{ii}^{(n)} = \infty
\end{equation*}

\textbf{Parte 2: ($\Leftarrow$)} Si el estado $i$ es transitorio, entonces $\sum_{n=0}^{\infty} P_{ii}^{(n)} < \infty$.

Si el estado $i$ es transitorio, entonces $f_{ii} < 1$. Sea $q = 1 - f_{ii} > 0$ la probabilidad de nunca regresar a $i$ partiendo de $i$.

El número de visitas a $i$ sigue una distribución geométrica modificada. La probabilidad de visitar exactamente $k$ veces el estado $i$ es:
\begin{equation*}
\mathbb{P}_i(N_i = k) = f_{ii}^{k-1} \cdot q = f_{ii}^{k-1} \cdot (1 - f_{ii})
\end{equation*}

Por lo tanto:
\begin{align*}
\mathbb{E}_i[N_i] &= \sum_{k=1}^{\infty} k \cdot f_{ii}^{k-1} \cdot (1 - f_{ii}) \\
&= (1 - f_{ii}) \sum_{k=1}^{\infty} k \cdot f_{ii}^{k-1} \\
&= (1 - f_{ii}) \cdot \frac{1}{(1 - f_{ii})^2} = \frac{1}{1 - f_{ii}} < \infty
\end{align*}

Como $\mathbb{E}_i[N_i] = \sum_{n=0}^{\infty} P_{ii}^{(n)}$ y $\mathbb{E}_i[N_i] < \infty$, concluimos que:
\begin{equation*}
\sum_{n=0}^{\infty} P_{ii}^{(n)} < \infty
\end{equation*}

\textbf{Conclusión:} Hemos demostrado ambas direcciones:
\begin{itemize}
\item Estado recurrente $\Leftrightarrow$ $\sum_{n=0}^{\infty} P_{ii}^{(n)} = \infty$
\item Estado transitorio $\Leftrightarrow$ $\sum_{n=0}^{\infty} P_{ii}^{(n)} < \infty$
\end{itemize}

\qed

\teorema{Si la cadena de Markov es irreducible y sus estados son recurrentes positivos, entonces la medida estacionaria $\pi$ existe y es única. Además:
\begin{equation*}
\pi_i = \frac{1}{m_i}, \quad i \in S
\end{equation*}
donde $m_i$ es el \textbf{tiempo medio de retorno} al estado $i$, es decir,
\begin{equation*}
m_i = E_i[T_i] \quad (\text{esperanza del tiempo hasta regresar a $i$ partiendo de $i$})
\end{equation*}
}

\textbf{Demostración:}

Fijemos un estado $i \in S$. Consideremos los \emph{ciclos de retorno} a $i$, definidos como las trayectorias que comienzan en $i$ y terminan en la siguiente visita a $i$.

\begin{itemize}
\item La longitud de un ciclo tiene la misma distribución que $T_i$, por lo que la longitud media es $m_i$.
\item En cada ciclo, el estado $i$ es visitado exactamente una vez más (la visita de cierre). Por tanto, el número medio de visitas a $i$ en un ciclo es $1$.
\end{itemize}

Sea $N_j$ el número de visitas al estado $j$ en un ciclo. Definimos
\begin{equation*}
\pi_j = \frac{E_i[N_j]}{m_i}
\end{equation*}

Esta definición nos da la fracción de tiempo que la cadena pasa en el estado $j$ durante un ciclo típico que comienza en $i$.

Como la cadena es irreducible, todos los estados se comunican, y por tanto esta definición no depende del estado inicial $i$ elegido. Además, se puede demostrar que:

\begin{enumerate}
\item $\sum_{j \in S} \pi_j = 1$ (normalización)
\item $\pi P = \pi$ (ecuación de balance)
\item $\pi_i = \frac{1}{m_i}$ para todo $i \in S$
\end{enumerate}

La unicidad se sigue del hecho de que el sistema de ecuaciones $\pi P = \pi$ junto con la condición de normalización tiene una única solución cuando la cadena es irreducible y finita.

\qed

\teorema{Sea $X_n$ una cadena de Markov $\{X_n\}_{n \geq 0}$ cuyos estados son irreducibles; recurrentes positivos y aperiódicos. Entonces:
\begin{equation*}
\lim_{n \to \infty} P^n(x,y) = \pi(y)
\end{equation*}}

\textbf{Demostración}:

La demostración se basa en el análisis del comportamiento asintótico de las potencias de la matriz de transición. Procederemos en varios pasos.

\textbf{Paso 1: Existencia y unicidad de la distribución estacionaria}

Por ser la cadena irreducible y recurrente positiva, sabemos del teorema anterior que existe una única distribución estacionaria $\pi$ tal que $\pi P = \pi$ y $\sum_{y} \pi(y) = 1$.

\textbf{Paso 2: Uso de la aperiodicidad}

Como los estados son aperiódicos, para cada estado $x$ existe un entero $N_x$ tal que para todo $n \geq N_x$, se tiene $P^n(x,x) > 0$. Esto significa que es posible regresar al estado $x$ en cualquier número suficientemente grande de pasos.

\textbf{Paso 3: Acoplamiento y tiempo de mezcla}

Definimos el \emph{tiempo de acoplamiento} $\tau$ como el primer momento en que dos copias independientes de la cadena, iniciando desde estados diferentes, se encuentran en el mismo estado.

Para estados irreducibles, recurrentes positivos y aperiódicos, se puede demostrar que:
\begin{equation*}
\mathbb{E}[\tau] < \infty
\end{equation*}

\textbf{Paso 4: Convergencia en variación total}

Sea $\mu_n^{(x)}$ la distribución de $X_n$ dado $X_0 = x$. Entonces:
\begin{equation*}
\mu_n^{(x)}(y) = P^n(x,y)
\end{equation*}

La distancia en variación total entre $\mu_n^{(x)}$ y $\pi$ está dada por:
\begin{equation*}
\|\mu_n^{(x)} - \pi\|_{TV} = \frac{1}{2}\sum_{y \in S} |P^n(x,y) - \pi(y)|
\end{equation*}

\textbf{Paso 5: Demostración de la convergencia}

Usando la técnica de acoplamiento, se puede demostrar que existe una constante $\rho < 1$ tal que:
\begin{equation*}
\|\mu_n^{(x)} - \pi\|_{TV} \leq C \rho^n
\end{equation*}

para alguna constante $C > 0$. Esto implica convergencia exponencial.

En particular, para cada estado $y$:
\begin{equation*}
|P^n(x,y) - \pi(y)| \leq 2\|\mu_n^{(x)} - \pi\|_{TV} \leq 2C \rho^n \to 0 \text{ cuando } n \to \infty
\end{equation*}

Por lo tanto:
\begin{equation*}
\lim_{n \to \infty} P^n(x,y) = \pi(y)
\end{equation*}

\textbf{Paso 6: Interpretación del resultado}

Este teorema nos dice que, independientemente del estado inicial $x$, la probabilidad de estar en el estado $y$ después de $n$ pasos converge a $\pi(y)$ cuando $n \to \infty$. Esto significa que la cadena "olvida" su condición inicial y converge a su distribución de equilibrio.

La velocidad de convergencia es exponencial con tasa $\rho$, lo que hace que la convergencia sea relativamente rápida en la práctica.

\qed

\textbf{Corolario:} Si además la cadena es finita, entonces la convergencia es uniforme en el estado inicial:
\begin{equation*}
\lim_{n \to \infty} \max_{x,y \in S} |P^n(x,y) - \pi(y)| = 0
\end{equation*}

\end{document}